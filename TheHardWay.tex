\documentclass[12pt]{article}
\usepackage{hyperref}
\hypersetup{
    colorlinks=true,
    linkcolor=blue,
    filecolor=magenta,
    urlcolor=cyan,
}
\begin{document}
\title{
    The hard way \\
    \large A beginners guide to programming
}
\author{Lilly H. St Claire}
\maketitle
\pagebreak
\tableofcontents
\pagebreak

\section{Preface}

To the uninitiated programming can be quite daunting.
From the obscure syntactical choices prevelant in
programming languages, to the high level problem
solving abundant in every programs goal, programming
can pose a unique level, and number of challenges
that wouldn't otherwise experience. In writing this
book I wish to teach my audience the skills required
to make long term projects.

I aim to achieve this by explaining in detail the
process required to make compilers of various levels.
From a project taking an afternoon, to week even
month long processes.

This book is aimed to be understood by \it both \rm
true beginners who have never touched code, as well
as programmers who have quite a portfolio. Anyone
who is interested in creating compilers should,
hopefully, be interested in this book.

\section{Environment}

Every beginner must first create an environment in which
you can edit, and run their programs. Due to this tutorial
being in Ocaml we will need to install its compiler by
following the instructions at
https://ocaml.org/docs/installing-ocaml.

Once that has been completed we will be using bash in this
tutorial, many of the commands will be similar (if not
identical in windows batch, rescourses can be found online).

\section{A bit of a mindfuck}

To begin with we will create a compiler for Brainfuck
an esoteric programming language (meaning a programming
language not designed for serious use) created by
Urbun M\"uller in 1993. The stated goal was to make the
programming language with the smallest possible compiler.

Brainfuck consists of an infinite number of containers
(known as "cells"), each cell is ordered by assigning
it a number. We store a number, known as the "pointer"
the user can only act on the cell corresponding to the
number held by the pointer. We can do the following
actions on the pointed cell.

\begin{itemize}
    \item{+ : add one to the cells stored number}
    \item{- : subtract one from the cells stored number}
    \item{, : read a letter from the user and store it in
        the cell}
    \item{. : print the letter stored by the current cell
        to the screen}
\end{itemize}

You should note that letters inputted by the user (and
outputted to the screen) are stored using an encoding
system known as ASCII. With ASCII each character
is assigned a number from 0 to 127, along with a few
control numbers (such as end of transmission, carriage
return, new line, etc.).

Additionally we can use {\tt > } to add one to the pointers
value, and {\tt < } to subtract one from it.

And finally we can repeat a series of actions by starting
a "{\tt loop }" These loops begin with {\tt [ } and end with
{\tt ] }. The program when running into the end of a loop
and if the pointed to cell is zero, then it will go back
to the start of the loop and do the actions from it to the
end statement again.

Any character that does not control the current cell,
pointer, or program is ignored by the compiler.

We will test our compiler with the following brainfuck
program.

\begin{verbatim}
[
    ,.          read a character into cell 0 and print it
    ----------  subtract 10 from that character
] if that character was 10, then 10-10 = 0 and the loop will end
note: 10 is the ASCII value for a new line (or enter key).
\end{verbatim}

\section{To begin programming}

Before we can build our first compiler a few fundamentals
must be understood. We will be programming in Ocaml, but
I implore you to adapt these programs to other languages.

In Ocaml a program consists of functions, and function calls.
For example:

\begin{verbatim}
let square x = x * x
\end{verbatim}

takes a number x and calculates x * x. We can call a function
in ocaml by simply typing it's name, and then the "arguments"
follow. See the following:

\begin{verbatim}
    square 10 (* evaluates to 100 *)
\end{verbatim}

(note that everything enclosed by {\tt (**) } is a comment
and ignored by the Ocaml compiler).

as before seen we can also apply operations to a value, these
are things like addition ({\tt a + b}), subtraction ({\tt a - b }),
multiplication ({\tt a * b}), and division ({\tt a / b}), which
are all rather self explanatory. Other more obscure operations
will be talked on when they come up.

\end{document}